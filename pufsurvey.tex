%\documentclass[paper]{ieicej}
%\documentclass[survey]{ieicej}% サーベイ論文
\documentclass[technicalreport]{ieicej} %今回のサーベイ論文の形式(多分)
%\documentclass[comment]{ieicej}% 解説論文
%\usepackage[dvips]{graphicx}
%\usepackage[dvipdfmx]{graphicx,xcolor}
\usepackage[fleqn]{amsmath}
\usepackage{newtxtext}% 英数字フォントの設定を変更しないでください
\usepackage[varg]{newtxmath}% % 英数字フォントの設定を変更しないでください
\usepackage{latexsym}
%\usepackage{amssymb}

\setcounter{page}{1}

\field{A}
\jtitle{PUFについてのサーベイ論文(仮)}
\etitle{Survey paper of PUF()}
\authorlist{
 \authorentry[okano.maiko.ol0@is.naist.jp]{岡野 舞子}{Maiko Okano}{NAIST}\MembershipNumber{2111062}
}
\affiliate[NAIST]{奈良先端科学技術大学院大学
\hskip1zw 〒630–0192 奈良県生駒市高山町8916–5}{
  Nara Institute of Science and Technology, 8916–5 Takayama-cho, Ikoma, Nara, Japan
}
%\affiliate[所属ラベル]{和文所属}{英文所属}
%\paffiliate[]{}
%\paffiliate[現在の所属ラベル]{和文所属}
\jalcdoi{???????????}% ← このままにしておいてください

\begin{document}
\begin{abstract}
  フィジカリー・アンクローナブル・ファンクション(PUF)とは,主に半導体技術を用いて作られた集積回路を大量生産した際に生じる,制御不能な製造ばらつき
  を利用してその個体にランダムな関数を作る技術のことである.この技術は,個体の識別に用いることで模造品の作成を防ぐだけでなく,制御不能な性質を利用することで
  暗号アルゴリズムに組み合わせて使うことも期待されている.本稿ではPUFについて調査した内容を,その発展の歴史を踏まえて述べる.
  (全部書き終えたら,ちゃんと書き直す)
\end{abstract}
\begin{keyword}
  %和文キーワード 4〜5語
  暗号ハードウェア,ハードウェアセキュリテイ,PUF
\end{keyword}
\begin{eabstract}
  %英文アブストラクト 100 words
\end{eabstract}
\begin{ekeyword}
  %英文キーワード
  Physically Unclonable Function
\end{ekeyword}
\maketitle

\section{はじめに}
研究背景,PUFの使いみち

\section{PUFとは}
\subsection{PUFの概要}
フィジカリー・アンクローナブル・ファンクション(Physicallly Unclonable Function: PUF)とは,
半導体技術を用いて作った集積回路などの大量生産された製造物に対して,製造時に生じる制御不能な製造ばらつきを
利用して,その個体に固有なランダム関数を作る技術のことを指す.
\subsection{PUFに求められる特性}
本節では,PUFに求められるセキュリティに関する特性を説明する.
要求される重要な特性の説明を通して,実際のPUFがどのような機能を果たすかの理解を深めることが狙いである.
しかし,本稿執筆時点(2021年7月2日)でPUFのセキュリティ要件及び試験方法は,ISO/IEC 20879にて議論中となっている.
そのため,ここでの内容は主に菅原\cite{sugatake}とMaes\cite{maes1}に基づいている.

まず,再現性・ユニーク性・耐クローン性の3つの特性について述べる.
この3つの特性は,PUFの機能の根幹を支えるとりわけ重要な特性である.
\subsubsection{再現性(Reproducibility)}
再現性とは,同一のPUFに同じチャレンジを繰り返し入力したとき,出力されるレスポンスのばらつきの小ささ,
つまり安定度の高さを表す.PUFのレスポンスにはノイズが含まれており,同じ個体に同じチャレンジを入力しても
出力されるレスポンスには差異が生じる.PUFの実用の観点から,同じ個体からのレスポンスは類似している必要があるため,
再現性は高いほうが良い.

再現性は,同一チップのPUF出力ビット列のハミング距離(Hamming Distance: HD),Intra-HDを評価指標としている.
このとき,Intra-HDの平均の理想値は0である.
また,この指標は,PUFのノイズの大きさを測るだけでなく,環境変化の影響を評価する場合にも用いる.
\subsubsection{ユニーク性(Uniqueness)}
ユニーク性とは,異なるPUFに同一のチャレンジを入力したとき,出力されるレスポンスの値の差の大きさを表す.異なるPUFのレスポンスの
値の差が小さい(類似している)場合,それらを同一であると誤判定してしまう可能性が高まる.そのため,
PUFの個体差,つまりユニーク性は高いほうがよい.

ユニーク性は,異なるチップのPUF出力ビット列のハミング距離,Inter-HDを評価指標としている.
このとき,Inter-HDの平均の理想値は0.5であり,この値に近いほどユニーク性は高い.
\subsubsection{耐クローン性(Unclonable)}
耐クローン性とは,PUFのチャレンジに対するレスポンスを再現するモデルの構築が不可能,あるいは困難であることを
表している.PUFを利用してある個体と別の個体を識別するためには,ある個体を模したクローンを作成できないという
前提条件が必要である.

この特性には2つの意味があり,1つは正規製造者であってもクローンを作ることができないこと(=製造者耐性),
もう1つはクローンの制作を試みる攻撃への耐性を持つ(クローンを作るための既知の攻撃が存在しない)ことである.


\section{PUFの分類}
\subsection{初期のPUF}
\subsubsection{Physical one-way functions}
PUFの始祖とされているのは,2001年にPappuが提案した物理的一方向性関数(Physical one-way functions: POWF)\cite{pappu}である.
背景としては,既存の数論に基づいた一方向性関数の課題に対する解決策として提案された.
POWFの簡略化した手順を以下に示す.
\begin{enumerate}
  \item 3次元の不規則な構造の媒体をトークンとして用意する.具体的には,極小のガラス玉をいくつか含んだエポキシ樹脂とされている.
  \item 上記の媒体にレーザー光を照射したものを,電化結合素子カメラで記録する.これによって,2次元のスペックルパターンを得る.
  \item スペックルパターンをガボール変換でフィルタリングし,鍵となるビット列(1次元)を生成する.
\end{enumerate}

\subsubsection{Sillicon PUF}
\label{SilliconPUF}
POWFは精密な光学機器で処理を行う必要があるため,アナログインターフェイスで使用するように作られている.
一方でGassendは2002年に,計算機やメモリと混載できるようにデジタルインターフェイスで動作する,半導体製のPUF(Sillicon PUF)を提案した\cite{gassend1}.
Sillicon PUFは前述のPOWFと異なり,暗号実装のハードウェア構成要素としてすぐに導入することができるため,
現在研究されているPUF構築の主要な種類となっている.
%このPUFは,各回路の遅延のばらつきを利用し,過渡応答を測定することでCRPsを生成する.
%過渡応答は,チャレンジによって刺激された経路上にあるIC内の配線やデバイスの遅延についての間接的な情報となる.
%この間接的な情報しかレスポンスとして与えられないことが,耐クローン性の根拠となっている.

Gassendが提案したPUF\cite{gassend1}の構成は,\ref{ROPUF}節で紹介する.
また,“PUF”という名前やPUFの定義が明記されたのは,この論文が最初である.

\subsection{遅延ベースのPUF}
\subsubsection{Arbiter PUF}
\ref{SilliconPUF}節で説明したSillicon PUFは回路の遅延が温度や電源電圧などの環境変化に敏感であるため,
PUFの信頼性を高めるためにはノイズが重大な問題である,とLeeは指摘を行った\cite{lee}.
そこでLimらは2004年にSillicon PUFの再現性を改善させたArbiter PUF\cite{lim}を実装した.
Arbiter PUFは差動構造(Arbiter)に基づいており,環境に起因するノイズに対するPUFの再現性を向上させることができる.
これは,PUFのレスポンスに対する絶対的な遅延値を測定する代わりに,2つの同一の遅延経路を比較し,
Arbiterを用いてデジタル情報を生成するためである.

%ここでアービターPUFの図を入れたい\label{APUF}
%前半,後半で分ける余裕があるかはわからん
Arbiter PUFの構造は前半と後半に分けられる.
最初に2つの入力から立ち上がり信号(=チャレンジ)が同時に入力され,
前半部ではチャレンジによって,2つの信号がどのような経路を伝搬するかが決定される.
図\ref{APUF}のとおり2つのMUXによってスイッチボックスが実装されており,これはチャレンジのビット数の分用意される.
スイッチボックスの動作は,チャレンジのビットが1のときは上下に流れる信号の経路を交差,0のときは直進となっている.

後半部では,どちらの信号が先に到達したかによって0か1のレスポンスを出力する.
ここでは1つのトランスペアレントラッチが使用されており,これは表\ref{apuf-transparent}が示すような動作を行う.
%表\label{apuf-tranparent}
\begin{table}[tb]
  \caption{和文キャプション}
  \label{apuf-transparent}
  %\ecaption{英文キャプション}
  \begin{center}
    \begin{tabular}{|l|l|l|}
      \Hline %% ←
      \multicolumn{2}{|c|}{\textbf{Inputs}} & \multicolumn{1}{c|}{\textbf{Outputs}}              \\
      \hline
      \textbf{G}                            & \textbf{D}                            & \textbf{Q} \\
      \hline
      0                                     & 0                                     & 0          \\
      \hline
      0                                     & 1                                     & 1          \\
      \hline
      1                                     & X                                     & No Change  \\
      \hline
      $\uparrow$                            & D                                     & d          \\
      \Hline %% ←
    \end{tabular}
  \end{center}
\end{table}
2つの立ち上がり信号が違うタイミングでDとGに入力された場合,Gに立ち上がり信号が入力された瞬間のDの値を常に出力し続ける.
つまり,Dに先に立ち上がり信号が到達した場合,その後Gに立ち上がり信号が到達することを意味するため,レスポンスは1となる.
逆にGに先に立ち上がり信号が到達した場合は,その時点でのDには0が入力されていることからレスポンスは0となる.

\subsubsection{Ring Ocillator PUF}
\label{ROPUF}

\subsection{メモリベースのPUF}
\subsubsection{SRAM PUF}
\subsubsection{Butterfly PUF}
\subsection{その他の特徴を持つPUF}
\subsection{Strong/Weak PUF}
\subsection{Controlled/Uncontrolled PUF}

\section{問題点や攻撃}
\subsection{暗号としてのPUF利用(仮)}

\subsection{PUFに対する攻撃}
\subsubsection{モデリング解析}
\section{まとめ}

%\ack %% 謝辞

%\bibliographystyle{sieicej}
%\bibliography{myrefs}
\begin{thebibliography}{99}% 文献数が10未満の時 {9}
  \bibitem{sugatake}
  菅原健,“暗号ハードウェアの研究開発動向:フィジカリー・アンクローナブル・ファンクション,”
  金融研究,vol.39,no.4,pp.25-54,Oct. 2020.
  \bibitem{maes1}
  R. Maes, Physically Unclonable Functions: Properties, Springer, Berlin, Heidelbelg, 2013.
  (DOI:10.1007/978-3-642-41395-7\_3)
  \bibitem{pappu}
  P.S. Ravikanth, “Physical One-Way Functions,” PhD thesis,
  Massachusetts Institute of Technology, March 2001.
  %\bibitem{gassend2}

  \bibitem{gassend1}
  B. Gassend, D. Clarke, M. van Dijk, S. Devadas, “Silicon physical random funcions,”
  CCS '02: Proceedings of the 9th ACM conference on Computer and communications security,
  pp.148-160, November 2002. (DOI: 10.1145/586110.586132)
  \bibitem{lee}
  J.W. Lee, D. Lim, B. Gassend, G.E. Suh, M. van Dijk, S. Devadas,
  “A Technique to Build a Secret Key in Integrated Circuits for Identification and Authentication Applications,”
  2004 Symposium on VLSI Circuits. Digest of Technical Papers (IEEE Cat. No.04CH37525),
  pp. 176-179, June 2004. (DOI: 10.1109/VLSIC.2004.1346548)
  %\bibitem{lim}
  %D. Lim, J.W. Lee, B. Gassend, G.E. Suh, M. van Dijk, S.Devadas,
  %“Extracting secret keys from integrated circuits,” IEEE Tran. Very Large Scale Integration (VLSI) Systems,
  %vol.13, no.10, pp.1200-1205, Oct. 2005.(DOI: 10.1109/TVLSI.2005.859470)
  \bibitem{lim}
  D. Lim, “Extracting Secret Keys from Integrated Circuits,” Master's Thesis,
  Massachusetts Institute of Technology, March 2004.

\end{thebibliography}

%\appendix
%\section{}

%\begin{biography}
% \profile{}{}{}
%\profile{会員種別}{名前}{紹介文}% 顔写真あり
%\profile*{会員種別}{名前}{紹介文}% 顔写真なし
%\end{biography}

\end{document}